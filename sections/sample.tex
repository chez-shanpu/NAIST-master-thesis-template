%! TEX root = ../main.tex
\documentclass[main]{subfiles}

\begin{document}

\section{サンプル}

\subsection{特定のルールを無視する}

以下の様に記述することで,ある行からある行までの間で特定のルールを無視できます.

% textlint-disable
\begin{lstlisting}[caption=textlint-filter-rule-commentsの使用方法,language=tex,breaklines=true]
    % textlint-disable ja-technical-writing/sentence-length
    これはとても長い長ーーーーーーーい文章なので、
    本来ならlintのルールに引っかかってしまいますが、
    このようにコメントを付加することでこの一文だけは
    lintの対象から除外されるので、特定のルールに
    どうしても引っかかってしまう・しかし修正することは
    難しい場合にご活用下さい
    % textlint-ensable ja-technical-writing/sentence-length
\end{lstlisting}
% textlint-enable

複数のルールを無視したい場合は,ルール名をカンマで区切って書きます.

\subsection{新規分割ファイルの追加}

LaTeX Workshopのためにマジックコメントを追加する必要があります.具体的には\lstlistingname\ref{code:newpage}のように記述します.

% textlint-disable
\begin{lstlisting}[caption=新章の追加テンプレート,language=tex,label=code:newpage]
    %! TEX root = ../main.tex
    \documentclass[main]{subfiles}

    \begin{document}

    \section{新セクション}

    これはサンプルの文章です

    \end{document}
\end{lstlisting}
% textlint-enable

\end{document}
